\chapter{Introducción}

En este capítulo introduciremos los principales problemas existentes en el contexto de minería de datos en el texto médico y pronpodremos una solución que se desarrollará a lo largo del documento.

\section{Texto sin formato}
Toda atención médica dispone de documentos que recogen toda la información relacionada con el paciente, enfermedad, y un seguimiento de ambos, así como los recursos a utilizar. En estos documentos suele haber una sección en la que el o la profesional en cuestión describe, en lo que podríamos denominar \textit{texto sin formato} todas estos factores. Debido a la carencia de formato, es difícil trabajar con dichas secciones, por lo que se suelen ignorar.

La idea es centrar nuestra atención en esas secciones de texto, con objeto de obtener la mayor cantidad de información posible y anexarla, ahora con formato, al documento del que provienen, enriqueciendo el informe y habilitando nuevas claves de búsqueda, así como mejorando el indexado de los documentos.


\section{Falta de datos}
Uno de los principales problemas a los que nos enfrentamos es la falta de \textit{datasets} o conjuntos de datos en los que estos documentos estén presentes. 

Se buscarán y agregarán tantas fuentes de datos como sea posible, se unificarán y se creará una herramienta que aproveche todos los datos disponibles públicamente para generar datos nuevos.

\section{Objetivos}
Dado este marco, describiremos en esta sección los objetivos de nuestro trabajo.

En primer lugar, se agregarán todas las fuentes de información públicas que nos provean con datos de comentarios médicos listos para su minería y análisis.

Utilizando todos estos datos, se hará una evaluación de las herramientas que ya existen en el estado del arte. Haremos una revisión de cómo se utilizan y del rendimiento de dichas herramientas. 

Sin embargo, para evaluar dichas herramientas, no utilizaremos los datos encontrados, sino que efectuaremos un flujo de trabajo alternativo. Utilizando técnicas de aprendizaje automático y generativo, crearemos un modelo que sea capaz de generar tantos comentarios médicos como sea necesario. La idea es suplir la carencia de datos con un modelo generativo, de forma que no se tenga que lidiar con aspectos de privacidad o licencia, ya que todos los comentarios serían generados de forma sintética. 

Si bien los comentarios son sintéticos, deben ser lo suficientemente convincentes como para que la evaluación de las herramientas sea fiel y rigurosa. Esto ofrece una herramienta para los desarrolladores de las herramientas que habilita a un mejor y más fructífero desarrollo, ya que se disponde de una cantidad, \textit{idealmente infinita} de comentarios.



