\appendix
\chapter{Apénice}

Incluiremos en este apéndice todos los bloques de código o cualquier otro tipo de contenido necesario para comprender la estructura del documento, pero evitando que interfiera con la lectura del mismo.

\begin{figure}[h]
	\centering
	
	\begin{minted}[breaklines, linenos]{python}
def regex_processing(text):
    # Remove capital letters surrounded by 0 or more `,` and a colon, i.e. the titles
    no_caps = re.sub(r',*([A-Z\s]+):', '', text)

    # Remove weirdly positioned commas. Find commas that dont have any letter before and some space after them.
    weird_commas = re.sub(r'(?<!\w),\s+', '', no_caps)
    
    # Remove commas that dont have spaces around them. (Commas should always have a trailing space after them)
    more_commas = re.sub(r'(?<!\s),(?!\s)', ' ', weird_commas)

    # Remove digits adyacent to dots or commas, as in enumerated lists.
    no_digits = re.sub(r'[\.,]*\d[\.,]+', ' ', more_commas)

    # Remove any other commas left behind the process. Particularly these cases: Hello. ,How are you?
    trailing_commas = re.sub(r'\s,(?=[A-Z\d])', '', no_digits)

    # Substitute any number of spaces for 1 single space.
    no_double_spaces = re.sub(r'\s+', ' ', trailing_commas)

    # Solve these problems: Hello .How are you? => Hello. How are you?
    final_text = re.sub(r'(?<!\s)\.(?!\s)', '. ', no_double_spaces)

    # Finally, strip the text from any trailing commas or white spaces.
    # The result is hopefully a clean version of the text, ready to be tokenized
    # and passed to the models.
    return final_text.strip(', ')
	\end{minted}

	\caption{Pipeline para el procesamiento de los comentarios de Medical Transcriptions}
	\label{code:pipeline-regex-mdtr}
\end{figure}


\section{Comentarios}
\label{sec:comments}

\begin{thm}
	\sffamily{
        Isolation of subcutaneous growth factor-alpha (SFL-alpha) in the stroma of lung carcinoma of the basolateral part of the rat lung and related tumors. We performed a prospective study to identify SFL-alpha-like growth factor-alpha (SFL-alpha) expression in the tissue and the body of lung carcinoma of the basolateral part of the rat lung and related tumors.
	}
\end{thm}
\begin{thm}
	\sffamily{
        Fibrous elastobrachial septal pressure syndrome in adults with delayed progressive disease. The prevalence of fibrous elastobrachial septal pressure syndrome in adult patients with delayed progressive disease is greater than 5\%. The duration of the disease is related to genetic factors, operative time, and duration of cure. The existence of this syndrome was examined in 103 children with the rare disease and in 17 patients with the rare disease.

	}
\end{thm}
\begin{thm}
	\sffamily{
        Intraperitoneal extracorporeal tone transplantation in children with renal echocardiography.
        \label{com:short}
	}
\end{thm}
\begin{thm}
	\sffamily{
        Biology of anaphylaxis in cardiomyopathy. Two case reports. Case reports of patients treated with biofilms for intracranial pressure syndrome and histologic abnormalities were reviewed. Anaphylaxis in patients with cardiomyopathy was described, and histologic abnormalities were found in six patients treated with biofilms for intracranial pressure syndrome. Treatment of cardiomyopathy requires strict care with regard to the pharmacological and hormonal effects of the biofilms.
	}
\end{thm}
\begin{thm}
	\sffamily{
        Antibody of the prostate with abnormal lumen density. Ultrastructural measurements were performed with respect to 30 age-matched patients on 11-year follow-up and for all clinical variables. Sixteen patients were identified by clinical examination as having abnormal lumen density; one patient was selected for transplant and one was selected for subsequent elective bone marrow transplantation. A total of 45 elective bone marrow transplantations were made. 
	}
\end{thm}