\chapter{Fundamentos de la minería de texto}
En este capítulo discutiremos algunas de las técnicas y términos más importantes a la hora de hablar de minería de texto, así como minería de datos en general, con objeto de que todas las consideraciones realizadas posteriormente queden claras.

En \textit{Text Mining Applied to Electronic Medical Records: A Literature Review}~\cite{textmining2015} se hace una revisión de los diferentes aspectos a tener en cuenta durante el procesamiento de textos médicos. Nos apoyaremos en gran medida en la estructura, contenidos y referencias de este artículo, que resume muy bien todo lo que necesitamos saber para resolver nuestro problema.

\section{Minería de datos}
La minería de datos es una rama de la informática que se dedica a encontrar tendencias y patrones en grandes volúmenes de información. Estas tendencias y patrones crean \textit{conocimiento} a partir de los datos, es decir: información estructurada desde los datos no estructurados. Esta información es muy valiosa y contribuye en las decisiones que se vayan a tomar o a monitorizar algunos aspectos que sean de vital importancia para el interesado. 

La minería de datos puede dividirse en un número de técnicas que funcionan de forma diferente en función del tipo de datos que tengamos y la información que busquemos.

\begin{enumerate}
    \item \textbf{Asociación}: esta técnica se centra en encontrar relaciones entre las distintas variables de nuestros datos, con objeto de encontrar muestras que sean estadísticamente dependientes. Una de las técnicas más utilizadas son las reglas de asociación, cuya salida tras el cálculo son un conjunto de reglas con antecedentes y consecuentes, muy fácilmente interpretables por cualquier persona, familiarizada o no con la ciencia de datos. \cite{associationrules1991}
    \item \textbf{Clasificación}: el proceso de clasificación trata de asignar una categoría a un conjunto de elementos que tengan algún aspecto en común. La clasificación en la minería de datos es una de las técnicas más utilizadas, ya que la naturaleza de gran parte de los datos responden bien a este método. \cite{Kumar2012ClassificationAF}
    \item \textbf{Agrupamiento}: también denominado \textit{clustering} trata de agrupar muestras que tengan características similares. A diferencia de la clasificación, aquí no tenemos una etiqueta o categoría a la que asignar las muestras, sino que las agrupamos \textit{a ciegas}, simplemente basándonos en alguna métrica para evaluar la distancia que haya entre un determinado par de muestras. \cite{Jain1999DataCA}
    \item \textbf{Predicción}: la predicción nos ayuda a encontrar tendencias entre variables, generalmente en datos con una componente temporal fuerte. \cite{han2012mining} Es común poder predecir si un paciente sufrirá una determinada enfermedad conociendo su historial médico, por ejemplo.
    \item \textbf{Identificación de patrones secuenciales}: Al igual que la predicción, se trabaja sobre datos con una componente temporal marcada. En este caso, se buscan patrones, es decir, conjuntos o cadenas de muestras que aparecen de forma frecuente en un orden concreto.
\end{enumerate}


\section{Minería de texto}
En esta sección, discutiremos los diferentes aspectos a tener en cuenta en la minería de textos en concreto, tras haber abordado el concepto de minería de datos en un ámbito más general.


\subsection{Términos}
Definiremos algunos de los términos más utilizados en esta disciplina, guiándonos principalmente por el trabajo de Kamran Kowsari, \textit{Text Classification Algorithms: A Survey}~\cite{Kowsari2019TextCA}.

\subsubsection{Tokens}
El término más esencial en minería de textos es \textit{token}. Un token es la mínima unidad en la que dividiremos un cuerpo de texto a la hora de analizarlo. Este elemento suele corresponderse con una palabra, que en el contexto de la mayoría de los idiomas corresponde con un conjunto de letras separado por espacios anterior y posteriormente. Esto da lugar a la creación de \textit{Tokenizers}, algoritmos que toman un cuerpo de texto como una cadena de caracteres muy larga, y devuelven un vector de palabras. Estos \textit{tokenizers } no han de tomar el espacio en blanco necesariamente ni exclusivamente como criterio divisor, aunque suele ser lo más común. Algunos de los \textit{tokenizers} más famosos son:

\begin{itemize}
    
    \item \textbf{Tokenizers de palabras}
    \begin{itemize}
        \item \textbf{Standard Tokenizer}: El Standard Tokenizer divide el texto en términos siguiendo los límites de las palabras según están definidos en el algoritmo \textit{Unicode Text Segmentation}. Funciona bien en general.
        \item \textbf{Letter Tokenizer}: divide el texto en términos cada vez que encuentra un carácter que no es una letra.
        \item \textbf{Whitespace Tokenizer}: Toma como criterio divisor el espacio en blanco.
        \item \textbf{Language Tokenizer}: Otros tipos de tokenizers adaptados a diferentes idiomas, como el inglés, que es el idioma más estudiado con diferencia, pero también otros idiomas con caracteres y reglas diferentes a aquellos basados en reglas occidentales, como el tailandés, o el chino.
    \end{itemize}
    \item \textbf{Tokenizers de palabras parciales}
    \begin{itemize}
        \item \textbf{N-Gram Tokenizer}: Este tokenizador incluye un parámetro adicional. Primero divide el texto con alguna de las reglas mencionadas anteriormente, y posteriormente, divide cada término del vector resultante en una ventana deslizante de $n$ elementos, de ahí \textit{N-Gram.} Por ejemplo: \textit{quick fox} devolvería $[$qu, ui, ic, ck$]$, $[$fo, ox$]$, dado un $n = 2$. Estos tokenizers también pueden utilizarse a nivel de párrafo, por lo que se devolverían pares de palabras, algo que puede ser muy útil para el análisis de \textit{dichos} o expresiones.
    \end{itemize}
    \item \textbf{Tokenizers de texto estructurado}
    \begin{itemize}
        \item \textbf{Pattern Tokenizer}: este tokenizer utiliza el patrón provisto como parámetro para la división de texto, utilizando expresiones regulares.
        \item \textbf{Simple Pattern Tokenizer}: este tokenizer utiliza el patrón provisto como parámetro para la división de texto, utilizando expresiones optimizadas para el patrón dado, lo que hace que funcione generalmente más rápido pero también será más específico.
    \end{itemize}
\end{itemize}

En resumen, un tokenizer es un algoritmo que divide el texto provisto siguiendo los criterios definidos por el usuario, devolviendo un vector con los elementos del texto divididos atendiendo a dichos criterios. Es una de las herramientas más esenciales en la minería de texto, ya que permite generar la mínima unidad de información a partir de la que se extraerá conocimiento.

\subsubsection{Palabras vacías}
\label{sec:stopwords}
Las palabras vacías o \textit{stopwords} son términos presentes en un idioma que sirven de apoyo para la formulación de oraciones pero que no poseen información en sí. Nos referimos a los artículos, determinantes, preposiciones, etc.

Estos términos son considerados como \textit{ruido} en el procesamiento de texto, por lo que lo más usual es disponer de un diccionario de términos vacíos y filtrar el texto original, eliminando dichos términos. De esta forma, nos quedamos con las palabras más importantes. Los símbolos de puntuación también se suelen considerar como ruido; si bien son esenciales para la comprensión y estructuración de texto para los humanos, suponen un detrimento para algoritmos de clasificación.

Sin embargo, esta operación es delicada y no siempre ofrecerá buenos resultados. Por ejemplo, si tratamos de inferir la intención de la oración \textit{No me gusta el fútbol} y pasamos previamente un filtro de palabras vacías, el texto resultante sería \textit{gusta fútbol}. Dados estos términos, se infiere que se está opinando de forma positiva acerca del tema \textit{fútbol}, cuando no es así.

Este caso particular está descrito en la literatura como \textit{negation handling}, en trabajos como \cite{Farooq2017NegationHI} o \cite{Ali2020ConventionalAS}. Aún así, hay muchos factores que se deben tener en cuenta antes de eliminar términos de una oración.



\subsubsection{Stemming y Lematización}
Stemming hace referencia a la gestión de palabras con prefijos o sufijos para su integración en una frase, como plurales (casa, casa\textbf{s}). Se trata de eliminar los posibles complementos añadidos con objeto de normalizar las palabras y que todas tengan la misma forma. En este caso también han de tenerse en cuenta las negaciones (típico, \textbf{a}típico).

La lematización va un paso más y trata de encontrar la raíz de las palabras, obteniendo una normalización más estricta. Un buen ejemplo son la conjugación de los verbos: de \textit{estudiando}, \textit{estudiante} o \textit{estudio} obtenemos \textit{estudi-}. \cite{Lemmatization2014} 


\subsubsection{Frecuencias: TF, IDF}
Uno de los datos más importantes a obtener de un texto es la frecuencia de palabras. Esta operación es tan simple como suena: contar cuántas veces aparece cada palabra y anotarlo en una estructura similar a un diccionario. Este término se conoce como \textit{Term Frequency} o TF. Estos valores suelen representarse en una escala logarítmica, con objeto de que las palabras muy dominantes no eclipsen a las menos frecuentes.

Del campo de teoría de la información \cite{information2001} conocemos que aquellos términos que aparezcan con una frecuencia muy alta poseerán menos información que aquellos que aparezcan menos. Como vimos en la sección \ref{sec:stopwords}, eliminamos las palabras vacías porque aparecían mucho. Es decir, un artículo como \textit{el} o una preposición como \textit{de} tendrían una frecuencia desproporcionada, cuando en realidad no aportan ninguna información. 


De forma similar, el valor \textit{Inverse Term Frequency}~\cite{Jones2004ASI} trata de abarcar esta frecuencia pero en un conjunto de documentos


\subsubsection{Bolsas de palabras}
Una bolsa de palabras es una versión reducida del documento procesado siguiendo algún criterio concreto, como la frecuencia de palabras.








