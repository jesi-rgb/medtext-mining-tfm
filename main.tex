\documentclass[12pt, a4paper, twoside]{report}

\usepackage[spanish]{babel}
\usepackage[utf8x]{inputenc}
\usepackage[T1]{fontenc}

% Para usar fuentes de letra distintas a la que viene por defecto es probable que te pida usar un compilador diferente a pdflatex, se puede cambiar fácilmente en el menú arriba a la izquierda.

%\usepackage{fontspec}
%\setmainfont{Arial}
\usepackage[table,xcdraw]{xcolor}
\usepackage{listings}

\usepackage{minted}
\usemintedstyle{pastie}

\usepackage{tikz}
\usepackage{pgfplots}
\pgfplotsset{compat=1.17}

\usepackage{filecontents}

\usepackage{wrapfig}

\usepackage{csquotes}
\usepackage{epigraph}

\usepackage{hyperref}
\hypersetup{hypertexnames=false, colorlinks=true, urlcolor=blue, linkcolor=black, citecolor=blue}
\usepackage{lscape}
\usepackage{subcaption}
\usepackage{amsmath}
\usepackage{graphicx}
\usepackage[colorinlistoftodos]{todonotes}
\usepackage[final]{pdfpages}
\usepackage[all]{nowidow}
\usepackage{multicol}
\usepackage{booktabs}
\usepackage{multirow}

\usepackage[]{algorithm2e}

\usepackage{titlesec}

\titleformat{\chapter}[block]
  {\scshape\huge}{\thechapter.}{1em}{\Huge}
\titlespacing*{\chapter}{0pt}{-19pt}{40pt}

\definecolor{graybox}{RGB}{235, 235, 235}
\newcommand{\jesitt}[1]{\colorbox{graybox}{\textcolor{red}{\texttt{#1}}}}

\renewcommand{\baselinestretch}{1.2}
\setlength{\parskip}{1em}
\renewcommand{\mkbegdispquote}[2]{\itshape}

%% Sets page size and margins
\usepackage[a4paper,top=2.5cm,bottom=2.5cm,left=2.5cm,right=2.5cm,marginparwidth=1.75cm]{geometry}

%Options: Sonny, Lenny, Glenn, Conny, Rejne, Bjarne, Bjornstrup
% \usepackage[Sonny]{fncychap}

\title{Análisis de textos médicos \\mediante NLP}
\author{Jesús Enrique Cartas Rascón}
% Puedes cambiar el profesor, la asignatura y demás en el archivo title/title.tex

\begin{document}
\begin{titlepage}

\newcommand{\HRule}{\rule{\linewidth}{0.5mm}} % Defines a new command for the horizontal lines, change thickness here

\center % Center everything on the page

%----------------------------------------------------------------------------------------
%	HEADING SECTIONS
%----------------------------------------------------------------------------------------
\quad\\[1.5cm]
%\textsc{\LARGE MSc Thesis}\\[1.5cm] % Name of your university/college
\textsc{\Large Trabajo Fin de Máster}\\[0.5cm] % Major heading such as course name

%----------------------------------------------------------------------------------------
%	TITLE SECTION
%----------------------------------------------------------------------------------------
\makeatletter
\HRule \\[0.4cm]
{ \huge \bfseries \@title}\\[0.4cm] % Title of your document
\HRule \\[1.5cm]
 
%----------------------------------------------------------------------------------------
%	AUTHOR SECTION
%----------------------------------------------------------------------------------------

\begin{minipage}{0.4\textwidth}
\begin{flushleft} \large
\emph{Autor:}\\
\@author % Your name
\end{flushleft}
\end{minipage}
~
\begin{minipage}{0.4\textwidth}
\begin{flushright} \large
\emph{Tutora:} \\
Rocío Romero Zaliz
% Uncomment the following lines if there's a co-supervisor
%\\[1.2em] % Supervisor's Name
%\emph{Co-Supervisor:} \\
%Dr. Adam Smith % second marker's name
\end{flushright}
\end{minipage}\\[3cm]
\makeatother


%----------------------------------------------------------------------------------------
%	DATE SECTION
%----------------------------------------------------------------------------------------

\vfill % Fill the rest of the page with whitespace

\end{titlepage}

\begin{abstract}
  En el ámbito de la medicina se almacena una gran cantidad de información relevante: desde valores numéricos correspondientes a signos vitales hasta texto plano que realiza un especialista para completar un informe. Muchas veces los datos guardados en el historial médico de un paciente, que no tiene una estructura determinada, son ignorados. Este proyecto propone recuperar texto médico sin formato para extraer nuevo conociemiento que pueda utilizarse para complementar la información estructurada y mejorar en la clasificación y tratamiento de los pacientes.
\end{abstract}


% Descomentar aquellos que sean necesarios
\tableofcontents
%\listoffigures
%\listoftables
%\listoflistings



% Contenido del documento
\chapter{Introducción}

En este capítulo introduciremos los principales problemas existentes en el contexto de minería de datos en el texto médico y pronpodremos una solución que se desarrollará a lo largo del documento.

\section{Texto sin formato}
Toda atención médica dispone de documentos que recogen toda la información relacionada con el paciente, enfermedad, y un seguimiento de ambos, así como los recursos a utilizar. En estos documentos suele haber una sección en la que el o la profesional en cuestión describe, en lo que podríamos denominar \textit{texto sin formato} todas estos factores. Debido a la carencia de formato, es difícil trabajar con dichas secciones, por lo que se suelen ignorar.

La idea es centrar nuestra atención en esas secciones de texto, con objeto de obtener la mayor cantidad de información posible y anexarla, ahora con formato, al documento del que provienen, enriqueciendo el informe y habilitando nuevas claves de búsqueda, así como mejorando el indexado de los documentos.


\section{Falta de datos}
Uno de los principales problemas a los que nos enfrentamos es la falta de \textit{datasets} o conjuntos de datos en los que estos documentos estén presentes. 

Se buscarán y agregarán tantas fuentes de datos como sea posible, se unificarán y se creará una herramienta que aproveche todos los datos disponibles públicamente para generar datos nuevos.

\section{Objetivos}
Dado este marco, describiremos en esta sección los objetivos de nuestro trabajo.

En primer lugar, se agregarán todas las fuentes de información públicas que nos provean con datos de comentarios médicos listos para su minería y análisis.

Utilizando todos estos datos, se hará una evaluación de las herramientas que ya existen en el estado del arte. Haremos una revisión de cómo se utilizan y del rendimiento de dichas herramientas. 

Sin embargo, para evaluar dichas herramientas, no utilizaremos los datos encontrados, sino que efectuaremos un flujo de trabajo alternativo. Utilizando técnicas de aprendizaje automático y generativo, crearemos un modelo que sea capaz de generar tantos comentarios médicos como sea necesario. La idea es suplir la carencia de datos con un modelo generativo, de forma que no se tenga que lidiar con aspectos de privacidad o licencia, ya que todos los comentarios serían generados de forma sintética. 

Si bien los comentarios son sintéticos, deben ser lo suficientemente convincentes como para que la evaluación de las herramientas sea fiel y rigurosa. Esto ofrece una herramienta para los desarrolladores de las herramientas que habilita a un mejor y más fructífero desarrollo, ya que se disponde de una cantidad, \textit{idealmente infinita} de comentarios.





\chapter{Fundamentos de la minería de texto}
En este capítulo discutiremos algunas de las técnicas y términos más importantes a la hora de hablar de minería de texto, así como minería de datos en general, con objeto de que todas las consideraciones realizadas posteriormente queden claras.

En \textit{Text Mining Applied to Electronic Medical Records: A Literature Review}~\cite{textmining2015} se hace una revisión de los diferentes aspectos a tener en cuenta durante el procesamiento de textos médicos. Nos apoyaremos en gran medida en la estructura, contenidos y referencias de este artículo, que resume muy bien todo lo que necesitamos saber para resolver nuestro problema.

\section{Minería de datos}
La minería de datos es una rama de la informática que se dedica a encontrar tendencias y patrones en grandes volúmenes de información. Estas tendencias y patrones crean \textit{conocimiento} a partir de los datos, es decir: información estructurada desde los datos no estructurados. Esta información es muy valiosa y contribuye en las decisiones que se vayan a tomar o a monitorizar algunos aspectos que sean de vital importancia para el interesado. 

La minería de datos puede dividirse en un número de técnicas que funcionan de forma diferente en función del tipo de datos que tengamos y la información que busquemos.

\begin{enumerate}
    \item \textbf{Asociación}: esta técnica se centra en encontrar relaciones entre las distintas variables de nuestros datos, con objeto de encontrar muestras que sean estadísticamente dependientes. Una de las técnicas más utilizadas son las reglas de asociación, cuya salida tras el cálculo son un conjunto de reglas con antecedentes y consecuentes, muy fácilmente interpretables por cualquier persona, familiarizada o no con la ciencia de datos. \cite{associationrules1991}
    \item \textbf{Clasificación}: el proceso de clasificación trata de asignar una categoría a un conjunto de elementos que tengan algún aspecto en común. La clasificación en la minería de datos es una de las técnicas más utilizadas, ya que la naturaleza de gran parte de los datos responden bien a este método. \cite{Kumar2012ClassificationAF}
    \item \textbf{Agrupamiento}: también denominado \textit{clustering} trata de agrupar muestras que tengan características similares. A diferencia de la clasificación, aquí no tenemos una etiqueta o categoría a la que asignar las muestras, sino que las agrupamos \textit{a ciegas}, simplemente basándonos en alguna métrica para evaluar la distancia que haya entre un determinado par de muestras. \cite{Jain1999DataCA}
    \item \textbf{Predicción}: la predicción nos ayuda a encontrar tendencias entre variables, generalmente en datos con una componente temporal fuerte. \cite{han2012mining} Es común poder predecir si un paciente sufrirá una determinada enfermedad conociendo su historial médico, por ejemplo.
    \item \textbf{Identificación de patrones secuenciales}: Al igual que la predicción, se trabaja sobre datos con una componente temporal marcada. En este caso, se buscan patrones, es decir, conjuntos o cadenas de muestras que aparecen de forma frecuente en un orden concreto.
\end{enumerate}


\section{Minería de texto}
En esta sección, discutiremos los diferentes aspectos a tener en cuenta en la minería de textos en concreto, tras haber abordado el concepto de minería de datos en un ámbito más general.


\subsection{Términos}
Definiremos algunos de los términos más utilizados en esta disciplina, guiándonos principalmente por el trabajo de Kamran Kowsari, \textit{Text Classification Algorithms: A Survey}~\cite{Kowsari2019TextCA}.

\subsubsection{Tokens}
El término más esencial en minería de textos es \textit{token}. Un token es la mínima unidad en la que dividiremos un cuerpo de texto a la hora de analizarlo. Este elemento suele corresponderse con una palabra, que en el contexto de la mayoría de los idiomas corresponde con un conjunto de letras separado por espacios anterior y posteriormente. Esto da lugar a la creación de \textit{Tokenizers}, algoritmos que toman un cuerpo de texto como una cadena de caracteres muy larga, y devuelven un vector de palabras. Estos \textit{tokenizers } no han de tomar el espacio en blanco necesariamente ni exclusivamente como criterio divisor, aunque suele ser lo más común. Algunos de los \textit{tokenizers} más famosos son:

\begin{itemize}
    
    \item \textbf{Tokenizers de palabras}
    \begin{itemize}
        \item \textbf{Standard Tokenizer}: El Standard Tokenizer divide el texto en términos siguiendo los límites de las palabras según están definidos en el algoritmo \textit{Unicode Text Segmentation}. Funciona bien en general.
        \item \textbf{Letter Tokenizer}: divide el texto en términos cada vez que encuentra un carácter que no es una letra.
        \item \textbf{Whitespace Tokenizer}: Toma como criterio divisor el espacio en blanco.
        \item \textbf{Language Tokenizer}: Otros tipos de tokenizers adaptados a diferentes idiomas, como el inglés, que es el idioma más estudiado con diferencia, pero también otros idiomas con caracteres y reglas diferentes a aquellos basados en reglas occidentales, como el tailandés, o el chino.
    \end{itemize}
    \item \textbf{Tokenizers de palabras parciales}
    \begin{itemize}
        \item \textbf{N-Gram Tokenizer}: Este tokenizador incluye un parámetro adicional. Primero divide el texto con alguna de las reglas mencionadas anteriormente, y posteriormente, divide cada término del vector resultante en una ventana deslizante de $n$ elementos, de ahí \textit{N-Gram.} Por ejemplo: \textit{quick fox} devolvería $[$qu, ui, ic, ck$]$, $[$fo, ox$]$, dado un $n = 2$. Estos tokenizers también pueden utilizarse a nivel de párrafo, por lo que se devolverían pares de palabras, algo que puede ser muy útil para el análisis de \textit{dichos} o expresiones.
    \end{itemize}
    \item \textbf{Tokenizers de texto estructurado}
    \begin{itemize}
        \item \textbf{Pattern Tokenizer}: este tokenizer utiliza el patrón provisto como parámetro para la división de texto, utilizando expresiones regulares.
        \item \textbf{Simple Pattern Tokenizer}: este tokenizer utiliza el patrón provisto como parámetro para la división de texto, utilizando expresiones optimizadas para el patrón dado, lo que hace que funcione generalmente más rápido pero también será más específico.
    \end{itemize}
\end{itemize}

En resumen, un tokenizer es un algoritmo que divide el texto provisto siguiendo los criterios definidos por el usuario, devolviendo un vector con los elementos del texto divididos atendiendo a dichos criterios. Es una de las herramientas más esenciales en la minería de texto, ya que permite generar la mínima unidad de información a partir de la que se extraerá conocimiento.

\subsubsection{Palabras vacías}
\label{sec:stopwords}
Las palabras vacías o \textit{stopwords} son términos presentes en un idioma que sirven de apoyo para la formulación de oraciones pero que no poseen información en sí. Nos referimos a los artículos, determinantes, preposiciones, etc.

Estos términos son considerados como \textit{ruido} en el procesamiento de texto, por lo que lo más usual es disponer de un diccionario de términos vacíos y filtrar el texto original, eliminando dichos términos. De esta forma, nos quedamos con las palabras más importantes. Los símbolos de puntuación también se suelen considerar como ruido; si bien son esenciales para la comprensión y estructuración de texto para los humanos, suponen un detrimento para algoritmos de clasificación.

Sin embargo, esta operación es delicada y no siempre ofrecerá buenos resultados. Por ejemplo, si tratamos de inferir la intención de la oración \textit{No me gusta el fútbol} y pasamos previamente un filtro de palabras vacías, el texto resultante sería \textit{gusta fútbol}. Dados estos términos, se infiere que se está opinando de forma positiva acerca del tema \textit{fútbol}, cuando no es así.

Este caso particular está descrito en la literatura como \textit{negation handling}, en trabajos como \cite{Farooq2017NegationHI} o \cite{Ali2020ConventionalAS}. Aún así, hay muchos factores que se deben tener en cuenta antes de eliminar términos de una oración.



\subsubsection{Stemming y Lematización}
Stemming hace referencia a la gestión de palabras con prefijos o sufijos para su integración en una frase, como plurales (casa, casa\textbf{s}). Se trata de eliminar los posibles complementos añadidos con objeto de normalizar las palabras y que todas tengan la misma forma. En este caso también han de tenerse en cuenta las negaciones (típico, \textbf{a}típico).

La lematización va un paso más allá y trata de encontrar la raíz de las palabras, obteniendo una normalización más estricta. Un buen ejemplo son la conjugación de los verbos: de \textit{estudiando}, \textit{estudiante} o \textit{estudio} obtenemos \textit{estudi-}. \cite{Lemmatization2014} 


\subsubsection{Frecuencias: TF, IDF}
Uno de los datos más importantes a obtener de un texto es la frecuencia de palabras. Esta operación es tan simple como suena: contar cuántas veces aparece cada palabra y anotarlo en una estructura similar a un diccionario. Este término se conoce como \textit{Term Frequency} o TF. Estos valores suelen representarse en una escala logarítmica, con objeto de que las palabras muy dominantes no eclipsen a las menos frecuentes.

Del campo de teoría de la información \cite{information2001} conocemos que aquellos términos que aparezcan con una frecuencia muy alta poseerán menos información que aquellos que aparezcan menos. Como vimos en la sección \ref{sec:stopwords}, eliminamos las palabras vacías porque aparecían mucho. Es decir, un artículo como \textit{el} o una preposición como \textit{de} tendrían una frecuencia desproporcionada, cuando en realidad no aportan ninguna información. 


De forma similar, el valor \textit{Inverse Document Frequency}~\cite{Jones2004ASI} trata de abarcar esta frecuencia pero en un conjunto de documentos, añadiendo la inversa de la frecuencia por documento. Esta métrica se utiliza mucho en conjunción con la TF, resultando en la TF-IDF, que trata de medir la relevancia de un término en un conjunto de documentos. Esto resulta en un cálculo tal que así:

\begin{equation}
    W(d, t) = TF(d, t) * log(\frac{N}{df(t)})
\end{equation}

donde $d$ es un documento del conjunto de documentos con cardinalidad $N$, $t$ es el término en concreto y $df(t)$ es el número de documentos que contienen el término $t$.


\subsubsection{Bolsas de palabras}
Conociendo el concepto de frecuencias de palabaras, una de las aplicaciones directas son las bolsas de palabras, que recogen en una estructura con forma de diccionario cada término y su frecuencia.

De esta forma, tenemos un \textit{ranking} para cada término. Esto se utiliza extensivamente en sistemas de recomendación, en donde una consulta provista por un usuario se compara con la bolsa de palabras del posible conjunto de documentos, y este conjunto va afinándose conforme se van comparando los conjuntos de palabras. El resultado es una búsqueda más refinada que devuelve documentos más relevantes con respecto a la consulta realizada.


\subsection{Técnicas}
En esta sección discutiremos algunas de las técnicas avanzadas más utilizadas en procesamiento y análisis de texto que además utilizaremos en nuestra implementación directa o indirectamente.

\subsubsection{\textit{Word Embeddings}}
Esta técnica esencialmente trata de convertir los diferentes términos en vectores de números reales, ya que esto los convierte en objetos matemáticos fáciles de comparar y procesar. Matemáticamente hablando corresponde con una representación de un espacio $n$-dimensional a un espacio vectorial continuo de menor tamaño, donde $n$ es el número total de términos presentes en todos los documentos.

Esta técnica ha sido estudiada en profundidad en varios proyectos:

\begin{itemize}
    \item \textbf{Word2Vec}: esta técnica trata de representar las palabras como vectores utilizando una red neuronal con dos capas, haciendo uso de una bolsa de palabras continua (CBOW) y el modelo del Skip Gram. \cite{Mikolov2013Word2Vec}
    \item \textbf{GloVe}: acrónimo de \textit{Global Vectors for Word Representation}, es una técnica muy similar a la \textit{Word2Vec}, con la particularidad de estar preentrenada en grandes corpus de texto, basados en Wikipedia y Gigaword. \cite{Pennington2014GloveGV}
    \item \textbf{FastText}: es una técnica desarrollada por Facebook. Esta técnica hace uso de la técnica de los $n$-gramas para su entrenamiento, obteniendo una representación de los términos mucho más granular. \cite{Bojanowski2017EnrichingWV}
    \item \textbf{Contextualized Word Representations}: esta técnica hace uso del contexto de las palabras para tratar de encontrar una representación y relación entre ellas. Esta técnica basa su funcionamiento en el uso de Long-Short Term Memory, un tipo de red neuronal recurrente muy utilizada en procesamiento de texto, en la que ahondaremos más en profundidad en secciones posteriores de este documento. \cite{Melamud2016context2vecLG}
\end{itemize}

\subsubsection{Reducción de dimensionalidad}
La reducción de dimensionalidad es una técnica que permite proyectar el espacio en el que se hallan nuestros datos en un subespacio de menor dimensionalidad, con objeto de facilitar el cálculo de las propiedades de dichos datos sin tener que utilizar todas sus características. Es común encontrar conjuntos de datos con un número de dimensiones muy alto, que hace inviable su estudio.

Las principales técnicas desarrolladas para reducir la dimensionalidad incluyen:

\begin{itemize}
    \item \textbf{Principal Component Analysis (PCA)}: PCA o análisis de componentes principales trata de encontrar un subespacio latente que represente a los datos encontrando aquellas variables que estén menos relacionadas y que maximicen la varianza, para conservar la mayor cantidad de variabilidad posible. \cite{Jolliffe:1986} 
    \item \textbf{Independent Component Analysis (ICA)}: es una técnica similar que trata de expresar los datos con transformaciones lineales. \cite{Hyvrinen2014TopographicIC}
    \item \textbf{Linear Discriminant Analysis (LDA)}: es otro método muy utilizado cuando los datos son de caracter categórico y no tienen una proporcion uniforme intraclase. \cite{LDA2009}
\end{itemize}

Toda esta familia de algoritmos resulta muy conveniente para \textit{comprimir} datos de alta dimensionalidad y extraer solo las \textbf{características principales} de los mismos. Se suelen usar en etapas de preprocesamiento, donde los datos resultantes se pasan a los algoritmos a entrenar, consiguiendo un mejor resultado y rendimiento en comparación con los datos sin preprocesar.

\subsubsection{Clasificación de texto}
Por último, abordaremos las principales técnicas para clasficar texto, ámbito importante en nuestro proyecto, así como una breve explicación de las mismas. Al igual que en las secciones anteriores, no es nuestro objetivo estudiarlas en profundida, pero más bien ofrecer una vista general del panorama en cuanto a esta tecnología con objeto de que el lector se familiarice con los términos.


Los principales algoritmos de clasificación de texto, entre otros, son los siguientes:

\begin{itemize}
    \item \textbf{Boosting y bagging}: boosting y bagging son dos algoritmos basados en lo que se denomina en la literatura como \textit{ensemble learning}. Esta técnica utiliza un gran número de modelos que funcionan muy bien para casos muy específicos pero no generalizan correctamente. La idea es que la respuesta conjunta de todos los modelos nos acerque a la respuesta correcta. Esta decisión puede hacerse mediante votos u otros métodos. \cite{Bauer2004AnEC}
    \item \textbf{Regresión logística}: la regresión logística es uno de los métodos de aprendizaje más simples, junto con la regresión linear. La regresión logística es una especialización de la regresión lineal, de forma que se utiliza una función logística para predecir categorías discretas, no continuas. \cite{CoxLogit1989}
    \item \textbf{Redes neuronales recurrentes}: por último, las redes neuronales recurrentes son una especialización de las redes neuronales en las que un subonjunto de las neuronas reciben su salida como una entrada, generándose ciclos de retroalimentación o \textit{feedback}. Estas redes funcionan especialmente bien con datos con patrones y componentes temporales, características especialmente destacables del lenguaje humano, así como de la música o vídeo. \cite{ZhouLSTM}
\end{itemize}


Existen otras muchas técnicas en clasificación de texto, como \textit{K Nearest Neighbours}~\cite{KNNXiao2007}, \textit{Naïve Bayes}~\cite{FrankNaive2006}, \textit{Support Vector Machines (SVM)}~\cite{SVMJoa1998}, árboles de decisión~\cite{DecisionTreeNoorman2018} o \textit{Random Forests}~\cite{breiman2001random}, entre muchas otras. 

\section{Estado del arte}

Por último, mencionaremos brevemente los tres proyectos más grandes de lenguaje generativo hasta la fecha, con objeto de entender qué es lo que hacen para poder integrar dichas técnicas en nuestro modelo.

\subsection{BERT}
BERT son las siglas en inglés de \textit{Bidirectional Encoder Representations from Transformers}~\cite{bertDevlin2019}. Es un modelo de lenguaje generativo basado en un encoder bidireccional como su nombre indica. Este modelo fue uno de los primeros en realmente conseguir una fluidez comunicativa convincente. Su arquitectura preentrenada permite, con una sola capa extra, craer modelos para tareas en casi cualquier ámbito, como responder preguntas o inferencia del lenguaje, sin necesidad de modificar particularmente su arquitectura interna.



\subsection{GPT-3 Open AI}
GPT son las siglas de \textit{Generative Pre-trained Transformer}~\cite{GPT3openAI2020} y 3 indica la versión. Este modelo trata de abarcar los problemas que otros transformers solían tener, como es la habilidad de un humano para continuar una tarea lingüística dado muy poco contexto o instrucciones. Escalando los modelos se logra mejorar significativamente la generalización del mismo y se descubre que no es necesario afinar los parámetros de los modelos, sino que esto puede corresponderse más con un problema de meta-aprendizaje.



\subsection{LaMDA}
LaMDA corresponde con las siglas de \textit{Language Model for Dialogue Applications}~\cite{LaMDAGoogle2020}, un proyecto de Google que compite de forma directa con los modelos antes mencionados. Al igual que los dos proyectos anteriores, LaMDA está basado en un transformer, una arquitectura de redes neuronales creada también por Google.~\cite{TransformerAshish2017} Este proyecto se creó con una aplicación en concreto: un \textit{chatbot} automático lo más natural posible, al que llamaron \textit{Meena}. Según las estadísticas de Google, Meena tiene casi el doble de capacidad de predicción e inferencia que el antiguo GPT-2, y se entrenó en 8 veces más datos. Es el buque insignia de la empresa.




\section{Datos: fuente y forma}
Los datos escogidos provienen del dataset \href{https://www.kaggle.com/chaitanyakck/medical-text}{Medical Text} publicado por Chaitanya Krishna Kasaraneni, y de \href{https://www.kaggle.com/tboyle10/medicaltranscriptions}{Medical Transcriptions}, publicado por Tara Boyle. La naturaleza de los mismos es ligeramente diferente así que explicaremos el proceso de preprocesamiento y unificación posteriormente.

En la Figura \ref{fig:preprocess-diagram} podemos ver un pequeño resumen del preprocesamiento que se acometerá a los datos. En la siguiente sección se detallan cada uno de estos pasos.

\subsection{Dataset: \textit{Medical Text}}
El dataset tiene formato \jesitt{.dat}, estructurado como un \jesitt{.tsv} (Tab Separated Values). La primera columna corresponde con una categoría determinada --ya que el dataset estaba diseñado para clasificación-- y la segunda columna contiene fragmentos de documentos médicos.

El dataset es en inglés, está anonimizado y los comentarios principalmente consisten en descripciones quirúrgicas o relacionadas con operaciones complejas. Podemos ver los dos primeros de nuestro conjunto de datos en los Comentarios \ref{com:com1} y \ref{com:com2}.

\begin{thm}
	\sffamily{
		Excision of limbal dermoids. We reviewed the clinical files of 10 patients who had undergone excision of unilateral epibulbar limbal dermoids. Preoperatively, all of the affected eyes had worse visual acuity (P less than .02) and more astigmatism (P less than .01) than the contralateral eyes. Postoperatively, every patient was cosmetically improved. Of the eight patients for whom both preoperative and postoperative visual acuity measurements had been obtained, in six it had changed minimally (less than or equal to 1 line), and in two it had improved (less than or equal to 2 lines). Surgical complications included persistent epithelial defects (40\%) and peripheral corneal vascularization and opacity (70\%). These complications do not outweigh the cosmetic and visual benefits of dermoid excision in selected patients. 
	}
	\label{com:com1}
\end{thm}
\begin{thm}
	\sffamily{
		Retained endobronchial foreign body removal facilitated by steroid therapy of an obstructing, inflammatory polyp. Oral and topical steroids were used to induce regression in an inflammatory, obstructing endobronchial polyp caused by a retained foreign body. The FB (a peanut half), which had been present for over six months, was then able to be easily and bloodlessly retrieved with fiberoptic bronchoscopy. 
	}
	\label{com:com2}
\end{thm}

El dataset está descompuesto en un archivo \jesitt{train.dat} y otro \jesitt{test.dat}. El archivo de entrenamiento contiene 14438 comentarios, y el de evaluación, 14442. En total, disponemos de 28880 comentarios.

Como podemos observar en las Figuras \ref{fig:avg_char_train} y \ref{fig:avg_tokens_train}, la distribución de los distintos elementos de nuestro dataset de entrenamiento aparenta estar normalmente distribuída, si eliminamos los valores atípicos a partir de más de 2500 tokens o más de 350 palabras por comentarios.

El número medio de caracteres por comentario es de 1230, y el número medio de tokens por comentario es de unos 180, correspondiéndose con las líneas amarillas en las figuras.

Como mencionábamos, se pueden apreciar algunos valores atípicos de comentarios particularmente largos. Esto, sin embargo, no es necesariamente malo en nuestro caso. En definitiva, cuanto más texto tengamos a nuestra disposición, mejor para el modelo.

\begin{figure}[h!]
	\centering
	\begin{subfigure}[t]{0.95\textwidth}
		\centering
		\includegraphics[width=.9\textwidth]{media/char_hist_train.pdf}
		\caption{Distribución del número de caracteres por comentario, en el conjunto de entrenamiento}
		\label{fig:avg_char_train}
	\end{subfigure}
	~

	\begin{subfigure}[t]{0.95\textwidth}
		\centering
		\includegraphics[width=.9\textwidth]{media/tokens_hist_train.pdf}
		\caption{Distribución del número de tokens por comentario en el conjunto de entrenamiento}
		\label{fig:avg_tokens_train}
	\end{subfigure}

	\caption{Visualización de la distribución de nuestro conjunto de entrenamiento}
	\label{fig:sum_train}
\end{figure}


\begin{figure}[h!]
	\centering
	\begin{subfigure}[t]{0.95\textwidth}
		\centering
		\includegraphics[width=.9\textwidth]{media/char_hist_test.pdf}
		\caption{Distribución del número de caracteres por comentario, en el conjunto de evaluación}
		\label{fig:avg_char_test_test}
	\end{subfigure}

	~

	\begin{subfigure}[t]{0.95\textwidth}
		\centering
		\includegraphics[width=.9\textwidth]{media/tokens_hist_test.pdf}
		\caption{Distribución del número de tokens por comentario en el conjunto de evaluación}
		\label{fig:avg_tokens_test}
	\end{subfigure}


	\caption{Visualización de la distribución de nuestro conjunto de evaluación}
	\label{fig:sum_test}
\end{figure}

Una media de casi 200 palabras por comentario con comentarios alcanzando las 500 corresponde con comentarios relativamente largos. Esto nos vendrá bien de cara al entrenamiento de nuestro modelo, para poder formar oraciones con más sentido.


\subsection{Dataset: \textit{Medical Transcriptions}}
Este dataset es en realidad una extracción de la página web \url{mtsamples.com}, donde se halla una respetable cantidad de trasncripciones médicas. La autora extrajo todos los comentarios, así como los diferentes metadatos que los acompañaban mediante \textit{web scraping} y los provee en la columna \jesitt{trasncription}.

Al igual que el anterior, este conjunto de datos se corresponde con informes de operaciones quirúrgicas en inglés, y de igual forma anonimizado.

En este caso, como podemos apreciar en la Figura \ref{fig:sum_mdtr}, las distribuciones son ligeramente asimétricas, predominando comentarios más cortos. Aún así, disponemos de comentarios excepcionalmente largos, con alrededor de 18000 caracteres.

\begin{figure}[h!]
	\centering
	\begin{subfigure}[t]{0.95\textwidth}
		\centering
		\includegraphics[width=.9\textwidth]{media/char_hist_mdtr.pdf}
		\caption{Distribución de caracteres en el dataset Medical Transcriptions}
		\label{fig:char_hist_mdtr}
	\end{subfigure}

	~
	\begin{subfigure}[t]{0.95\textwidth}
		\centering
		\includegraphics[width=.9\textwidth]{media/token_hist_mdtr.pdf}
		\caption{Distribución de palabras en el dataset Medical Transcriptions}
		\label{fig:token_hist_mdtr}
	\end{subfigure}

	\caption{Visualización del dataset Medical Transcriptions}
	\label{fig:sum_mdtr}
\end{figure}



\section{Preprocesamiento}
\label{sec:preprocess}
En esta sección, describiremos el preprocesamiento acometido en cada uno de los datsets. Provienen de fuentes diferentes así que cada uno recibirá un trato diferente, con objeto de normalizar y unificar el formato de todos de cara al entrenamiento.




\subsection{Dataset: Medical Text}
Este conjunto de datos, siendo específicamente texto, el formato, ortografía y en general formato de los archivos es muy bueno. Simplemente hemos de eliminar las categorías adjuntas a cada comentario, para obtener una lista de comentarios crudos en sí. Por lo demás, los comentarios carecen de problemas de formato, codificación o cualquier otra cosa que pudiera interferir con el proceso de entrenamiento. Probablemente el autor ya hiciera esto por nosotros antes de publicarlo.


\subsection{Dataset: Medical Transcriptions}
En el caso de las trancripciones médicas, la tarea es considerablemente más compleja. El conjunto de datos proviene de la página web \url{mtsamples.com}, como especificamos anteriomente. La autora efectuó un proceso de \textit{web scraping} para obtener toda la información y recogerla en el archivo \jesitt{.csv}. 

Esto facilita las cosas, pero desde luego los comentarios deben ser tratados en profundidad antes de poder pasarlos a cualquier modelo. Los trazos de formato en HTML se dejan entrever en los comentarios con signos de puntuación o tabulaciones fuera de lugar, apreciables en el Comentario \ref{com:mt-before}, así que debemos arreglarlo previo entrenamiento.

Para ello, se ha hecho un fuerte uso de expresiones regulares, y se ha creado un \textit{pipeline} para procesar todo el texto a la vez.

El pipeline elimina todas las posibles trazas o residuos que hubieran quedado del \textit{web scraping}. Podemos ver el pipeline diseñado en la Figura \ref{code:pipeline-regex-mdtr} del Apéndice.

El resumen del proceso es eliminar signos de puntuación mal colocados, eliminar títulos o cabeceras de secciones de la página web, sustituir múltiples espacios por uno solo o eliminar los números de listas enumeradas (1., 2., etc). Finalmente, se añaden las etiquetas que vemos en la Figura \ref{fig:preprocess-diagram}. Se explicará su funcionamiento en la sección de la experimentación.

El resultado es un texto muy limpio y claro, mucho más apto para la fase de entrenamiento.

Podemos ver una comparativa del antes (Comentario \ref{com:mt-before}) y el después (Comentario \ref{com:mt-after}) del preprocesamiento de un determinado comentario de nuestro dataset. Los comentarios han sido truncados debido a su longitud.


\begin{thm}
	\sffamily{
		\small{
		SUBJECTIVE:,  This 23-year-old white female presents with complaint of allergies.  She used to have allergies when she lived in Seattle but she thinks they are worse here.  In the past, she has tried Claritin, and Zyrtec.  Both worked for short time but then seemed to lose effectiveness.  She has used Allegra also.  She used that last summer and she began using it again two weeks ago. [...]
		}
	}
	\label{com:mt-before}
\end{thm}


\begin{thm}
	\sffamily{
		\small{
		$<|$ BOS $|>$This 23-year-old white female presents with complaint of allergies. She used to have allergies when she lived in Seattle but she thinks they are worse here. In the past, she has tried Claritin, and Zyrtec. Both worked for short time but then seemed to lose effectiveness. She has used Allegra also. She used that last summer and she began using it again two weeks ago. [...] $<|$ EOS $|>$}
	}
	\label{com:mt-after}
\end{thm}


Tras haber preprocesado y cribado todos los elementos que pudieran suponer un problema a la hora de entrenar nuestro modelo, lo que obtenemos es un dataset unificado con un total de 33846 comentarios, sumando un total de 7537697 palabras con una media de 222 tokens y 1482 caracteres por comentario. 


Vemos además, en la Figura \ref{fig:wordcloud} una nube de palabras de todo el conjunto de datos. 

\begin{figure}[h]
	\centering
	\includegraphics[width=.9\textwidth]{media/wordcloud_full_data.png}
	\caption{Wordcloud de todo el conjunto de datos}
	\label{fig:wordcloud}
\end{figure}

Se aprecia que las palabras más comunes son \textit{patient}, \textit{case}, o \textit{treatment}, junto con \textit{study} o \textit{performed}. Esta nube de palabras se ha calculado en el texto preprocesado, pero el preprocesamiento solo ha tenido en cuenta cuestiones de formato, mayoritariamente. De ahí que términos como \textit{used}, \textit{using} o \textit{use} aparezcan a la vez. No hemos hecho stemming ni lemmatization porque nos interesa el contenido de las oraciones tal y como está, que es como el modelo fue diseñado para ser entrenado.

Disponiendo de este dataset, estamos listos para poder entrenar y ajustar nuestro modelo para que genere comentarios muy similares a los presentes en nuestro conjunto de datos.
\chapter{Experimentación: modelos \\ y entrenamiento}

En este capítulo usaremos todos los conceptos vistos en los capítulos anteriores para justificar las elecciones de los diferentes modelos, hablaremos de las características principales de los mismos y finalmente entraremos en la fase del entrenamiento y los resultados de dichos modelos.







\section{Experimentación}


\section{Resultados}







% \appendix
\chapter{Apénice}

Incluiremos en este apéndice todos los bloques de código o cualquier otro tipo de contenido necesario para comprender la estructura del documento, pero evitando que interfiera con la lectura del mismo.

\begin{figure}[h]
	\centering
	
	\begin{minted}[breaklines, linenos]{python}
def regex_processing(text):
    # Remove capital letters surrounded by 0 or more `,` and a colon, i.e. the titles
    no_caps = re.sub(r',*([A-Z\s]+):', '', text)

    # Remove weirdly positioned commas. Find commas that dont have any letter before and some space after them.
    weird_commas = re.sub(r'(?<!\w),\s+', '', no_caps)
    
    # Remove commas that dont have spaces around them. (Commas should always have a trailing space after them)
    more_commas = re.sub(r'(?<!\s),(?!\s)', ' ', weird_commas)

    # Remove digits adyacent to dots or commas, as in enumerated lists.
    no_digits = re.sub(r'[\.,]*\d[\.,]+', ' ', more_commas)

    # Remove any other commas left behind the process. Particularly these cases: Hello. ,How are you?
    trailing_commas = re.sub(r'\s,(?=[A-Z\d])', '', no_digits)

    # Substitute any number of spaces for 1 single space.
    no_double_spaces = re.sub(r'\s+', ' ', trailing_commas)

    # Solve these problems: Hello .How are you? => Hello. How are you?
    final_text = re.sub(r'(?<!\s)\.(?!\s)', '. ', no_double_spaces)

    # Finally, strip the text from any trailing commas or white spaces.
    # The result is hopefully a clean version of the text, ready to be tokenized
    # and passed to the models.
    return final_text.strip(', ')
	\end{minted}

	\caption{Pipeline para el procesamiento de los comentarios de Medical Transcriptions}
	\label{code:pipeline-regex-mdtr}
\end{figure}

\bibliographystyle{unsrt}
\bibliography{bibs/bibliography}
\addcontentsline{toc}{chapter}{Bibliografía}

\end{document}